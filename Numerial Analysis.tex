\documentclass{article}
  \renewcommand\thesubsection{\alph{subsection}}
  \usepackage{listings}
  \usepackage{enumerate}
  \usepackage{amsmath, amsthm, amssymb}
  \usepackage{CJK}
  \usepackage{graphicx}
  \usepackage{subfloat}
  \usepackage{ulem}
  \usepackage[framed,numbered,autolinebreaks,useliterate]{mcode}

    \title{Homework 1}                   %———总标题
    \author{Hu Bin\\7297853}
    \date{}
 \begin{document}
    \maketitle                                  % —— 显示标题
     %\tableofcontents                               %—— 制作目录(目录是根据标题自动生成的)


    \section{Problem 1}                      %——一号子标题,星号*取消section数字编号

      \subsection{}                      %——二号子标题 
      In matlab, I typed the following codes.\\
        \lstset{language=Matlab}
        \begin{lstlisting}
          x = pi;
          y = sin(10^4*x)/x
        

        \end{lstlisting}
        y = -1.5460e-13\\
        The numerical arpproximation error is -1.5460e-13. It's because computers store numbers with finit precision,
        and pi in Matlab is not equal to the real $\pi$ which is with infinite precision.

        \subsection{}
        Expand sin function as Maclawrin series, we have
    
        \begin{align*}
          f(x) =& \frac{sin(10^4x)}{x}\\
               =& 10^4-\frac{10^{12} x^2}{3!}+o(x^4)
        \end{align*}
        When $x=10^{-10}$, $ f(10^{-10})=10^4-\frac{10^{-8}}{3!}+o(x^4) $ . 
        The numerical ecaluation error is about $1.6667 \times 10^{-9} $.

    \section{Problem 2}
      \subsection{}
        Matlab code as followed:
        \newpage
        \begin{lstlisting}[caption = \mcode{my_diff.m}]
          function out=my_diff(f,x,h)
            f1=f(x+h);
            f2=f(x-h);
            out=(f1-f2)./(2*h);
          end
        \end{lstlisting}
      \subsection{}
        \begin{lstlisting}[caption = \mcode{Problem 2 answer}]
          my_func=@(x)exp(x)
          h=logspace(-15,-1,1000);
          y=my_diff(my_func,0,h);
          loglog(h,y)
          xlabel("h")
          ylabel("f'(x)")
        \end{lstlisting}

        \begin{figure}[ht]
          \centering
          \includegraphics[width = .8\textwidth]{hw2_2.eps}
          \caption{this is a figure demo}
          \label{fig:label}
        \end{figure}

 \end{document}
